\documentclass[12pt]{article}

\usepackage[normalem]{ulem}										% Used for strikethrough text with the \sout{} command
\usepackage{amsmath}
\usepackage{amssymb}
\usepackage{geometry}
\usepackage{graphicx}
\usepackage{tikz}
\usepackage{epstopdf}
\usepackage{caption}
\usepackage{subcaption}
\PassOptionsToPackage{hyphens}{url}\usepackage{hyperref}		% Everything before \usepackage helps with the wrapping of urls
\usepackage{pdfcomment}
\usepackage{bookmark}
\makeatletter
\renewcommand\@seccntformat[1]{}							% http://tex.stackexchange.com/questions/33696/no-section-numbers-but-still-have-pdf-bookmarks-with-hyperref
\makeatother

\geometry{
letterpaper,tmargin=1in,bmargin=1in,lmargin=1in,rmargin=1in
}

\hypersetup{
colorlinks,linkcolor=blue,citecolor=violet,urlcolor=magenta
}

\setlength\parindent{0pt} % Removes all indentation from paragraphs

\begin{document}
\title{Smart Vibes Project Log}
\author{Ezequiel Juarez Garcia}
\date{Starting on January 23, 2017}
\maketitle

\newpage

\section{Monday, January 23, 2017}
\begin{itemize}
\item Worked on first presentation and report on Google Drive.
\item Converted the Google Doc into a Latex doc
\end{itemize}

\section{Tuesday, January 24, 2017}
\begin{itemize}
\item I was chosen as team leader by Joshua due to his busy schedule impeding him from leading.
\item Performed presentation 1.
\item Jordan bought a Feather for about \$40 in order to start programming it. The reason he bought it on his own is because we don't know when the parts will get ordered.
\item Submitted first presentation and report to Git repo. Also submitted two peer assessments of Positive Resonance and Intellisense teams.
\end{itemize}

\textbf{To do list:}
\begin{itemize}
\item \sout{Fix system level design overview.} (Completed on 1/27/2017)
\item \sout{Create a Github private repo to host Design 2 group work.} (Completed on 1/27/2017)
\end{itemize}

\section{Thursday, January 26, 2017}
\begin{itemize}
\item Submitted peer assessments for Poly Builders, Team Concrete, and Phoenix Designs.
\end{itemize}

\section{Friday, January 27, 2017}
\begin{itemize}
\item Created a private Githup repo named Design 2 for our group to host code and other work.
\item Begin working with Adafruit Feather that Jordan ordered and received
\item Learning objectives for Adafruit Feather:
	\begin{itemize}
	\item Write up a simple LED program
	\item Use SPI
	\item Connect using the built-in Wifi module
	\end{itemize}
\item Setup procedure for Adafruit Feather:
	\begin{enumerate}
	\item Insert the header pins on a breadboard and place the Feather on top of the pins. No soldering required.
	\item Went to the following website: \url{https://learn.adafruit.com/adafruit-feather-m0-wifi-atwinc1500/}.
	\item Download and install Arduino IDE v1.6.4+.
	\item Go to \texttt{File} $\rightarrow$ \texttt{Preferences}. Type in \url{https://adafruit.github.io/arduino-board-index/package_adafruit_index.json} into the \textbf{Additional Boards Manager URLs} box.
	\item Navigate to \texttt{Tools} $\rightarrow$ \texttt{Boards} $\rightarrow$ \texttt{Boards Manager}. Install \textbf{Arduino SAMD Boards} version 1.6.8
	\item Install the \textbf{Adafruit SAMD} package version 1.0.13.
	\item Quit and reopen Arduino IDE. After restart, new boards will be listed on \texttt{Tools} $\rightarrow$ \texttt{Boards}. Select the appropriate board.
	\item Install drivers (Windows only). Link: \url{https://github.com/adafruit/Adafruit_Windows_Drivers/releases/download/1.0.0.0/adafruit_drivers.exe}.
	\end{enumerate}

\end{itemize}

\section{Friday, February 3, 2017}
Revised the functional diagram and bill of materials.

\section{Monday, February 6, 2017}
\begin{itemize}
\item Worked on presentation 2.
\item Worked on report 2.
\end{itemize}

\section{Thursday, February 9, 2017}
\begin{itemize}
\item Met together with team during class time, even though it was cancelled, to work on programming.
\item Josh began designing the casing in SolidWorks.
\item Marc was given the task of learning the cloud access aspect of this project.
\item Jordan and I will program the sensor.
\item Can SPI communicate with parallel devices at the same time like I2C? \texttt{SPI devices communicate in full duplex mode using a master-slave architecture with a single master. The master device originates the frame for reading and writing. Multiple slave devices are supported through selection with individual slave select (SS) lines.}
\item Does the wifi chip not allow other communication to happen simultaneously? \texttt{.}
\end{itemize}

\section{Friday, February 17, 2017}
\begin{itemize}
\item Go together with Marc and Jordan. Jordan and I looked how to get the wifi working. Marc took charge of learning how to push stuff to Azure cloud.
\item I managed to scan for networks and connect to an access point set up by Marc with the Feather microcontroller.
\item We are left with the question of how to connect the Feather to MS Azure and push data to it.
\end{itemize}

\section{Friday, February 24, 2017}
\begin{itemize}
\item Met together with my team at 11am to work on the presentation due on Monday.
\end{itemize}

\section{Friday, March 3, 2017}
\begin{itemize}
\item Began programming the other Feather that we received.
\item Follow the wifi tutorial on this \href{https://learn.adafruit.com/adafruit-feather-m0-wifi-atwinc1500/using-the-wifi-module}{Adafruit} page.
\item Go to \textbf{Sketch} $\rightarrow$ \textbf{Include Library} $\rightarrow$ \textbf{Manage Libraries} and search for \textbf{wifi101}. Install the latest version.
\item Go to the Adafruit page for the \href{https://learn.adafruit.com/lsm303-accelerometer-slash-compass-breakout/assembly-and-wiring}{LSM303DLHC} to learn how to wire it up and write code for it.
\item Installed the LSM303DLHC throgh the Library Manager and the Adafruit Sensor library from the git \href{https://github.com/adafruit/Adafruit_Sensor}{repo}.
\end{itemize}

\section{Thursday, March 9, 2017}
\begin{itemize}
\item Googled \textbf{arduino mongodb example} for help!
\end{itemize}

% I'm going to start taking better logs from this point forward!
\section{Tuesday, March 14, 2017}
\begin{itemize}
\item Visited this \href{https://www.adafruit.com/product/3010}{Adafruit} page on everything about the Adafruit Feather M0 WiFi - ATSAMD21 + ATWINC1500 microcontroller.
\item Googled \textsl{Atsamd21} to learn more about the chip architecture. Clicked on an \href{http://www.atmel.com/products/microcontrollers/arm/sam-d.aspx}{Atmel} page to learn more. More specifically, the Feather features an ATSAMD21G18A ARM Cortex-M0+ processor. The Feather also includes an Atmel ATWINC1500 Wi-Fi network controller.
\item The datasheet for the ATSAMD21 can be found \href{https://cdn-learn.adafruit.com/assets/assets/000/030/130/original/atmel-42181-sam-d21_datasheet.pdf?1453847579}{here} and a programming guide for the ATWINC1500 can be found \href{https://cdn-learn.adafruit.com/assets/assets/000/030/129/original/atmel-42418-software-programming-guide-for-atwinc1500-wifi-using-samd21-xplained-pro_userguide.pdf?1453847486}{here}.
\item Googled what ASF is; the short answer is the following: the Atmel Software Framework (ASF) is a MCU software library providing a large collection of embedded software for Atmel flash MCUs. More info on the ASF can be found \href{http://www.atmel.com/tools/avrsoftwareframework.aspx}{here}.
\item Googled what an mcu is; it stand for microncontroller unit.
\item Visited this \href{https://learn.adafruit.com/adafruit-feather-m0-wifi-atwinc1500/pinouts}{Adafruit} page to learn more on the pinouts of the Feather.
\item Googled what UART is; go to this \href{https://learn.sparkfun.com/tutorials/serial-communication/uarts}{Sparkfun} page to learn more. Basically, the universal asynchronous receiver/transmitter (UART) is a block of circuitry responsible for implementing serial communication.
\end{itemize}

\textbf{To do:} Look into EagleCAD and Fritzing software.

\section{Wednesday, March 15, 2017}
\begin{itemize}
\item Go to this \href{https://www.adafruit.com/product/1120}{Adafruit} page to learn more aobut the LSM303DLHC compass and accelerometer.
\item Go to this \href{https://learn.adafruit.com/lsm303-accelerometer-slash-compass-breakout/assembly-and-wiring}{Adafruit} page for the assembly and wiring of the LSM303.
\item Downloaded the Adafruit Fritzing Library from this \href{https://github.com/adafruit/Fritzing-Library}{GitHub} repo in order to get access to the LSM303DLHC, Feather, and Adalogger diagrams.
\item Connected the accelerometer to the microcontroller. The Fritzing object is under the \textsf{fritzing\_objects} folder in the Design 2 Github repo.
	\begin{itemize}
	\item Currently having trouble with detecting the sensor!
	\item Fixed the above problem. Use a lower value for the pullup resistors, like 2.2 k$\Omega$.!
	\end{itemize}
\item \href{https://cdn-shop.adafruit.com/datasheets/LSM303DLHC.PDF}{Datasheet} of the LSM303DLHC accelerometer and compass.
\item Visit this \href{https://learn.adafruit.com/lsm303-accelerometer-slash-compass-breakout/calibration}{Adafruit} page to learn more about the calibration of the accelerometer and compass. There is a calibration tool that can be found \href{http://www.varesano.net/blog/fabio/freeimu-magnetometer-and-accelerometer-calibration-gui-alpha-version-out}{here}.
\item Calibration steps in Windows PC:
	\begin{itemize}
	\item Go to the this \href{https://github.com/mjs513/FreeIMU-Updates/wiki/04.-FreeIMU-Calibration}{GitHub} page for more detailed instructions.
	\item Download the this \href{http://bazaar.launchpad.net/~fabio-varesano/freeimu/trunk/tarball/49?start_revid=49}{tarball} of the source files.
	\item Download \href{https://www.python.org/ftp/python/2.7.13/python-2.7.13.msi}{Python 2.7.13}.
	\item \textbf{keep it going...}
	\end{itemize}
\end{itemize}

\textbf{To do:}
\begin{itemize}
\item Look up the power-saving features of the LSM303DLHC (accelerometer).
\item Perform interrupts using accelerometer.
\item Write data to the SD card with the Adalogger breakout board.
\end{itemize}

\section{Thurssday, March 16, 2017}
Had a group talk with Harish, here are the notes.
\begin{itemize}
\item Study the example code for the accelerometer.
\end{itemize}

\section{Friday, March 17, 2017}
\begin{itemize}
\item Cloned the Azure Docs \href{https://github.com/Microsoft/azure-docs}{GitHub} repo.
\item Cloned the Azure IoT Starter Kit \href{https://github.com/Azure-Samples/iot-hub-c-m0wifi-getstartedkit}{GitHub} repo. The Microsoft page on the same subject can be found \href{https://azure.microsoft.com/en-us/resources/samples/iot-hub-c-m0wifi-getstartedkit/}{here}.
\item More Microsoft IoT repos can be found \href{https://github.com/ms-iot}{here}.
\item The Windows Remote Arduino Experience app can be found \href{https://www.microsoft.com/store/apps/9nblggh2041m}{here}. With it, you can control an Arduino compatible device with a Windows 10 computer. The GitHub repo can be found \href{https://github.com/ms-iot/remote-wiring#notes-on-wifi-and-ethernet}{here}.
\item The Windows Remote Arduino Samples repo can be found \href{https://github.com/ms-iot/windows-remote-arduino-samples}{here}.
\end{itemize}

\textbf{To do:} Take a look at the last three bullet points.

\section{Sunday, March 19, 2017}
\begin{itemize}
\item Looked up what analog-to-digital converter channels meant. It means the number of inputs the ADC can select from.
\item Looked up what direct memory access (DMA) channels were. Short for direct memory access, a technique for transferring data from main memory to a device without passing it through the CPU. Computers that have DMA channels can transfer data to and from devices much more quickly than computers without a DMA channel can. To learn more go to the \href{https://en.wikipedia.org/wiki/Direct_memory_access}{Wikipedia} page.
\end{itemize}

\section{Monday, March 20, 2017}
\begin{itemize}
\item Read more on timer interrupts. I don't undertand much of what the says on timer counters.
\item Googled timer interrupts on ATSAMD21G and stumbled across this \href{http://www.avdweb.nl/arduino/libraries/samd21-timer.html}{site}. This guy made his own library to abstract interrupts. I will to download and install them.
	\begin{itemize}
	\item Downloaded \href{https://github.com/avandalen/avdweb_SAMDtimer}{SAMDtimer} library.
	\item Also clone the Arduino ZeroTimer library.
	\item Could not get the example to work!
	\end{itemize}
\item Visited this \href{https://learn.adafruit.com/using-atsamd21-sercom-to-add-more-spi-i2c-serial-ports/muxing-it-up}{Adafruit} website with very useful SERCOM info.
\item Cloned this \href{https://github.com/drewfish/arduino-ZeroRegs}{repo} that helps you print out the low-level configuration registers for the Arduino Zero and similar boards.
\end{itemize}

\section{Tuesday, March 21, 2017}
\begin{itemize}
\item SERCOM 3 is used for I2C.
\item Tried to understand the SAM D21 datasheet.
\item  Tried to get the yesterday's library to work.
	\begin{itemize}
	\item Cloned \href{https://github.com/avandalen/avdweb_SAMDtimer}{avdweb SAMDtimer} library.
	\item Cloned \href{https://github.com/adafruit/Adafruit_ZeroTimer}{Arduino ZeroTimer} library.
	\item Cloned \href{https://github.com/adafruit/Adafruit_ASFcore}{Adafruit ASFcore} library.
	\item The program compiled but I don't know what pin the PWM signal is being sent to. =(
	\end{itemize}
\item This \href{https://forum.arduino.cc/index.php?topic=457720.0}{Arduino Forum} post tells you what the equivalen \textsf{cli} and \textsf{sei} commands are for SAM D21 boards.
\item Cloned \href{https://github.com/maxbader/arduino_tools}{Arduino tools} repo.
\item Followed this \href{https://medium.com/@nebsp/smoothly-changing-a-timers-frequency-on-the-arduino-zero-1f6cd285369a#.xl6i5bu18}{guide} related to timer interrupts and tested all the digital output pins.
\item \url{https://github.com/maxbader/arduino_tools/blob/master/libraries/timer_zero_tc_counter/timer_zero_tc_counter.ino}
\item \url{http://www.gammon.com.au/interrupts}
\end{itemize}

\section{Tuesday, March 21, 2017}
\begin{itemize}
\item Run \texttt{pip install xbee pyserial}.
\item Go this \href{https://groups.google.com/forum/#!topic/python-xbee/jdyQ5ev9_vg}{Google Gruops} post to learn how to fix the following error: \textsf{``Unrecognized response packet with id byte {0}''.format(data[0]))
KeyError: 'Unrecognized response packet with id byte \textbackslash x90'}
\end{itemize}

\section{To Do List}
\begin{itemize}
\item Look into EagleCAD and \sout{Fritzing software}.
\item Look up the power-saving features of the LSM303DLHC (accelerometer).
\item Perform interrupts using accelerometer.
\item Write data to the SD card with the Adalogger breakout board.
\item Take a look at the last three bullet points for March 17, 2017.
\item Read up on DMA.
\item Left off on page 28 of \textsl{atmel-42181-sam-d21\_datasheet.pdf}.
\item \url{https://learn.adafruit.com/adafruit-feather-m0-wifi-atwinc1500/adapting-sketches-to-m0}
\item \url{https://forum.arduino.cc/index.php?topic=332275.15}
\item \url{https://gist.github.com/jdneo/43be30d85080b175cb5aed3500d3f989}
\item \url{https://github.com/arduino/ArduinoCore-samd/issues/136}
\item \url{https://docs.microsoft.com/en-us/azure/iot-hub/iot-hub-adafruit-feather-m0-wifi-kit-arduino-get-started}
\end{itemize}

\section{Sunday, March 26, 2017}
\begin{itemize}
\item Read an article by \href{http://www.ni.com/white-paper/3807/en/}{NI} on measuring vibrations with accelerometers.
\item The datasheet of the accelerometer can be found \href{https://cdn-shop.adafruit.com/datasheets/LSM303DLHC.PDF}{here}.
\item They are basically three components to ground motion: longitudinal or radial (X direction), transversal (Y direction), and vertical (Z direction). \textbf{We should write the XYZ directions clearly outside the sensor housing.}
\item Read the following pages:
	\begin{itemize}
	\item \url{http://en.allexperts.com/q/Construction-Contractors-1093/2008/5/PPV-s-G.htm}
	\item \url{http://vibrationdamage.com/Vibration_standards.htm}
	\end{itemize}
\item Currently working on vibration sensor sketch. \textbf{I will postpone the FFT for later.}	
\item I'm moving on to writing data to the SD card. The real-time clock (RTC) chip running on the Adalogger FeatherWing is the PCF8523.
\item \textbf{Configure the Adalogger:}
	\begin{itemize}
	\item Install the RTClib by Adafruit library through the library manager in the IDE.
	\item Set the right time using the example. FYI, it takes about 7 seconds for compilation and uploading of example.
	\end{itemize}
	
\item Adalogger resources:
	\begin{itemize}
	\item \href{http://www.nxp.com/documents/data_sheet/PCF8523.pdf}{RTC datasheet}
	\item \href{https://learn.adafruit.com/adafruit-adalogger-featherwing/using-the-real-time-clock}{Adafruit} page on using the RTC.
	\end{itemize}

\end{itemize}

\section{Sunday, March 26, 2017}
\begin{itemize}
\item Scrapped the SD library in favor of the SdFat library over at \url{https://github.com/greiman/SdFat}.
\end{itemize}


\section{Tuesday -- April 4, 2017}
\begin{itemize}
\item Watched this \href{https://www.youtube.com/watch?v=wLlsVXBaQLQ}{YouTube} on how to wire up a speaker with an Arduino.
\item Followed this \href{https://docs.microsoft.com/en-us/azure/iot-hub/iot-hub-adafruit-feather-m0-wifi-kit-arduino-get-started}{MS Azure} IoT tutorial of the Adafruit Feather M0.
\end{itemize}

\section{Friday -- April 7, 2017}
\begin{itemize}
\item The speaker we have is a 4$\Omega$ 2W speaker that can handle up to around 700 mA.
\end{itemize}

\section{Tuesday -- April 11, 2017}
\begin{itemize}
\item Worked vigorously to get the remote monitoring sketch to work.
\item Updated the WiFi101 firmware of the Feather to version WINC1501 Model B (19.5.2)
\end{itemize}

\end{document}
