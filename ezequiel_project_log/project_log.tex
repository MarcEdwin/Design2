\documentclass[12pt]{article}

\usepackage[normalem]{ulem}										% Used for strikethrough text with the \sout{} command
\usepackage{amsmath}
\usepackage{amssymb}
\usepackage{geometry}
\usepackage{graphicx}
\usepackage{tikz}
\usepackage{epstopdf}
\usepackage{caption}
\usepackage{subcaption}
\PassOptionsToPackage{hyphens}{url}\usepackage{hyperref}		% Everything before \usepackage helps with the wrapping of urls
\usepackage{pdfcomment}

\geometry{
letterpaper,tmargin=1in,bmargin=1in,lmargin=1in,rmargin=1in
}

\hypersetup{
colorlinks,linkcolor=blue,citecolor=violet,urlcolor=magenta
}

\setlength\parindent{0pt} % Removes all indentation from paragraphs

\begin{document}
\title{Smart Vibes Project Log}
\author{Ezequiel Juarez Garcia}
\date{Starting on January 23, 2017}
\maketitle

\newpage

\section{Monday, January 23, 2017}
\begin{itemize}
\item Worked on first presentation and report on Google Drive.
\item Converted the Google Doc into a Latex doc
\end{itemize}

\section{Tuesday, January 24, 2017}
\begin{itemize}
\item I was chosen as team leader by Joshua due to his busy schedule impeding him from leading.
\item Performed presentation 1.
\item Jordan bought a Feather for about \$40 in order to start programming it. The reason he bought it on his own is because we don't know when the parts will get ordered.
\item Submitted first presentation and report to Git repo. Also submitted two peer assessments of Positive Resonance and Intellisense teams.
\end{itemize}

\textbf{To do list:}
\begin{itemize}
\item \sout{Fix system level design overview.} (Completed on 1/27/2017)
\item \sout{Create a Github private repo to host Design 2 group work.} (Completed on 1/27/2017)
\end{itemize}

\section{Thursday, January 26, 2017}
\begin{itemize}
\item Submitted peer assessments for Poly Builders, Team Concrete, and Phoenix Designs.
\end{itemize}

\section{Friday, January 27, 2017}
\begin{itemize}
\item Created a private Githup repo named Design 2 for our group to host code and other work.
\item Begin working with Adafruit Feather that Jordan ordered and received
\item Learning objectives for Adafruit Feather:
	\begin{itemize}
	\item Write up a simple LED program
	\item Use SPI
	\item Connect using the built-in Wifi module
	\end{itemize}
\item Setup procedure for Adafruit Feather:
	\begin{enumerate}
	\item Insert the header pins on a breadboard and place the Feather on top of the pins. No soldering required.
	\item Went to the following website: \url{https://learn.adafruit.com/adafruit-feather-m0-wifi-atwinc1500/}.
	\item Download and install Arduino IDE v1.6.4+.
	\item Go to \texttt{File} $\rightarrow$ \texttt{Preferences}. Type in \url{https://adafruit.github.io/arduino-board-index/package_adafruit_index.json} into the \textbf{Additional Boards Manager URLs} box.
	\item Navigate to \texttt{Tools} $\rightarrow$ \texttt{Boards} $\rightarrow$ \texttt{Boards Manager}. Install \textbf{Arduino SAMD Boards} version 1.6.8
	\item Install the \textbf{Adafruit SAMD} package.
	\item Quit and reopen Arduino IDE. After restart, new boards will be listed on \texttt{Tools} $\rightarrow$ \texttt{Boards}. Select the appropriate board.
	\item Install drivers (Windows only). Link: \url{https://github.com/adafruit/Adafruit_Windows_Drivers/releases/download/1.0.0.0/adafruit_drivers.exe}.
	\end{enumerate}

\end{itemize}

\section{Friday, February 3, 2017}
Revised the functional diagram and bill of materials.

\section{Monday, February 6, 2017}
\begin{itemize}
\item Worked on presentation 2.
\item Worked on report 2.
\end{itemize}

\section{Thursday, February 9, 2017}
\begin{itemize}
\item Met together with team during class time, even though it was cancelled, to work on programming.
\item Josh began designing the casing in SolidWorks.
\item Marc was given the task of learning the cloud access aspect of this project.
\item Jordan and I will program the sensor.
\item Can SPI communicate with parallel devices at the same time like I2C? \texttt{SPI devices communicate in full duplex mode using a master-slave architecture with a single master. The master device originates the frame for reading and writing. Multiple slave devices are supported through selection with individual slave select (SS) lines.}
\item Does the wifi chip not allow other communication to happen simultaneously? \texttt{.}
\end{itemize}

\section{Friday, February 17, 2017}
\begin{itemize}
\item Go together with Marc and Jordan. Jordan and I looked how to get the wifi working. Marc took charge of learning how to push stuff to Azure cloud.
\item I managed to scan for networks and connect to an access point set up by Marc with the Feather microcontroller.
\item We are left with the question of how to connect the Feather to MS Azure and push data to it.
\end{itemize}

\section{Friday, February 24, 2017}
\begin{itemize}
\item Met together with my team at 11am to work on the presentation due on Monday.
\end{itemize}

\section{Friday, March 3, 2017}
\begin{itemize}
\item Began programming the other Feather that we received.
\item Follow the wifi tutorial on this \href{https://learn.adafruit.com/adafruit-feather-m0-wifi-atwinc1500/using-the-wifi-module}{Adafruit} page.
\item Go to \textbf{Sketch} $\rightarrow$ \textbf{Include Library} $\rightarrow$ \textbf{Manage Libraries} and search for \textbf{wifi101}. Install the latest version.
\item Go to the Adafruit page for the \href{https://learn.adafruit.com/lsm303-accelerometer-slash-compass-breakout/assembly-and-wiring}{LSM303DLHC} to learn how to wire it up and write code for it.
\item Installed the LSM303DLHC throgh the Library Manager and the Adafruit Sensor library from the git \href{https://github.com/adafruit/Adafruit_Sensor}{repo}.
\end{itemize}

\section{Thursday, March 9, 2017}
\begin{itemize}
\item Googled \textbf{arduino mongodb example} for help!
\end{itemize}

% I'm going to start taking better logs from this point forward!
\section{Tuesday, March 14, 2017}
\begin{itemize}
\item Visited this \href{https://www.adafruit.com/product/3010}{Adafruit} page on everything about the Adafruit Feather M0 WiFi - ATSAMD21 + ATWINC1500 microcontroller.
\item Googled \textsl{Atsamd21} to learn more about the chip architecture. Clicked on an \href{http://www.atmel.com/products/microcontrollers/arm/sam-d.aspx}{Atmel} page to learn more. More specifically, the Feather features an ATSAMD21G18 ARM Cortex-M0+ processor. The Feather also includes an Atmel ATWINC1500 Wi-Fi network controller.
\item The datasheet for the ATSAMD21 can be found \href{https://cdn-learn.adafruit.com/assets/assets/000/030/130/original/atmel-42181-sam-d21_datasheet.pdf?1453847579}{here} and a programming guide for the ATWINC1500 can be found \href{https://cdn-learn.adafruit.com/assets/assets/000/030/129/original/atmel-42418-software-programming-guide-for-atwinc1500-wifi-using-samd21-xplained-pro_userguide.pdf?1453847486}{here}.
\item Google what ASF is; the short answer is the following: the Atmel Software Framework (ASF) is a MCU software library providing a large collection of embedded software for Atmel flash MCUs. More info on the ASF can be found \href{http://www.atmel.com/tools/avrsoftwareframework.aspx}{here}.
\item Googled what an mcu is; it stand for microncontroller unit.
\item \textbf{Keep reading this} Visited this \href{https://learn.adafruit.com/adafruit-feather-m0-wifi-atwinc1500/pinouts}{Adafruit} page to learn more on the pinouts of the Feather.
\item Googled what UART is; go to this \href{https://learn.sparkfun.com/tutorials/serial-communication/uarts}{Sparkfun} page to learn more. Basically, the universal asynchronous receiver/transmitter (UART) is a block of circuitry responsible for implementing serial communication.
\end{itemize}

\textbf{To do:} Look into EagleCAD and Fritzing software.

\end{document}
