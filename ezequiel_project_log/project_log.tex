\documentclass[12pt]{article}

\usepackage[normalem]{ulem}										% Used for strikethrough text with the \sout{} command
\usepackage{amsmath}
\usepackage{amssymb}
\usepackage{geometry}
\usepackage{graphicx}
\usepackage{tikz}
\usepackage{epstopdf}
\usepackage{caption}
\usepackage{subcaption}
\PassOptionsToPackage{hyphens}{url}\usepackage{hyperref}		% Everything before \usepackage helps with the wrapping of urls
\usepackage{pdfcomment}

\geometry{
letterpaper,tmargin=1in,bmargin=1in,lmargin=1in,rmargin=1in
}

\hypersetup{
colorlinks,linkcolor=blue,citecolor=violet,urlcolor=magenta
}

\setlength\parindent{0pt} % Removes all indentation from paragraphs

\begin{document}
\title{Smart Vibes Project Log}
\author{Ezequiel Juarez Garcia}
\date{Starting on January 23, 2017}
\maketitle

\newpage

\section*{Monday, January 23, 2017}
\begin{itemize}
\item Worked on first presentation and report on Google Drive.
\item Converted the Google Doc into a Latex doc
\end{itemize}

\section*{Tuesday, January 24, 2017}
\begin{itemize}
\item I was chosen as team leader by Joshua due to his busy schedule impeding him from leading.
\item Performed presentation 1.
\item Jordan bought a Feather for about \$40 in order to start programming it. The reason he bought it on his own is because we don't know when the parts will get ordered.
\item Submitted first presentation and report to Git repo. Also submitted two peer assessments of Positive Resonance and Intellisense teams.
\end{itemize}

\textbf{To do list:}
\begin{itemize}
\item \sout{Fix system level design overview.} (Completed on 1/27/2017)
\item \sout{Create a Github private repo to host Design 2 group work.} (Completed on 1/27/2017)
\end{itemize}

\section*{Thursday, January 26, 2017}
\begin{itemize}
\item Submitted peer assessments for Poly Builders, Team Concrete, and Phoenix Designs.
\end{itemize}

\section*{Friday, January 27, 2017}
\begin{itemize}
\item Created a private Githup repo named Design 2 for our group to host code and other work.
\item Begin working with Adafruit Feather that Jordan ordered and received
\item Learning objectives for Adafruit Feather:
	\begin{itemize}
	\item Write up a simple LED program
	\item Use SPI
	\item Connect using the built-in Wifi module
	\end{itemize}
\item Setup procedure for Adafruit Feather:
	\begin{enumerate}
	\item Insert the header pins on a breadboard and place the Feather on top of the pins. No soldering required.
	\item Went to the following website: \url{https://learn.adafruit.com/adafruit-feather-m0-wifi-atwinc1500/}.
	\item Download and install Arduino IDE v1.6.4+.
	\item Go to \texttt{File} $\rightarrow$ \texttt{Preferences}. Type in \url{https://adafruit.github.io/arduino-board-index/package_adafruit_index.json} into the \textbf{Additional Boards Manager URLs} box.
	\item Navigate to \texttt{Tools} $\rightarrow$ \texttt{Boards} $\rightarrow$ \texttt{Boards Manager}. Install \textbf{Arduino SAMD Boards} version 1.6.8
	\item Install the \textbf{Adafruit SAMD} package.
	\item Quit and reopen Arduino IDE. After restart, new boards will be listed on \texttt{Tools} $\rightarrow$ \texttt{Boards}. Select the appropriate board.
	\item Install drivers (Windows only). Link: \url{https://github.com/adafruit/Adafruit_Windows_Drivers/releases/download/1.0.0.0/adafruit_drivers.exe}.
	\end{enumerate}

\end{itemize}


\end{document}
