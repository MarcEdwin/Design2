
\documentclass[12pt]{article}


\usepackage{amsmath}
\usepackage{amssymb}
\usepackage{geometry}
\usepackage{graphicx}
\usepackage{tikz}
\usepackage{epstopdf}
\usepackage{caption}
\usepackage{subcaption}
\usepackage{hyperref}
\usepackage[author={Harish}]{pdfcomment}

\geometry{letterpaper,tmargin=1in,bmargin=1in,lmargin=1in,rmargin=1in}
\hypersetup{
colorlinks,linkcolor=blue,citecolor=violet,urlcolor=magenta
}



\begin{document}

\begin{center}
\large
\topskip0pt
\vspace*{\fill}
Design and development of \\
\textbf{$<$project name here$>$}

\vspace{2cm}

Design 2 - Final Report \\
Florida Polytechnic University \\
Spring 2017\\

\vspace{2cm}
submitted by:\\
Team Name\\
Member 1 (percent contribution)\\
Member 2 (percent contribution)\\
Member 3 (percent contribution)\\
Member 4 (percent contribution)\\
Member 5 (percent contribution)\\
\vspace*{\fill}

\end{center}




\newpage

\tableofcontents



\newpage

\section{Introduction}

The introduction of your report should  briefly describe your initial project statement, activities completed during the design process, highlights of the prototyping process, your final product, and concluding remarks.  The entire report should be 5 pages or less, single spaced, have 1-inch margins, and use Cambria 11 point font.  Body text of the report should be justified.   Each new section should contain a 4-5 sentence summary of the information contained in the section. 

 Figures and tables that are included in the body of the report should be numbered and called out in the text referencing them.  All the figures should be sufficiently captioned, such that a reader who is just looking at the figures must obtain a ``reasonable'' amount of information about your product. 


Your team may make use of appendices to provide supplemental material.  Reference each appendix number and section directly after any related text.  

In this report, and in any technical writing in general, make sure to be as brief as possible to give all the necessary information, no longer, no shorter. 



\section{Customer Needs, Objectives and Team Interpretation}

Brief 4 to 5 sentence introduction of the section.  

\subsection{Customer needs}
Describe the design problem and customer needs initially stated by the client.  

\subsection{Design Objectives}
In this section discuss the initial set of engineering requirements stated by the customer or interpreted through customer needs statements.  Design objectives will be more mathematical and take the form of engineering requirements.  These differ from customer needs in that they are not as subjective.  


\subsection{Team Interpretation}
Use this section to describe how your team interpreted the customer need and design objectives.  Also discuss the final set of requirements for your project.  Often times the objectives and needs will change will be refined throughout the design process.  Use this space to tell the reader the final list of needs and objectives your product is designed to meet. 



\section{Concept Generation and Analysis of Alternatives}

Again, begin each section with a 4-5 sentence summary of the information contained within the section.  Use this section to discuss the concepts your team generated and how your team decided which concepts were more appropriate.  



\subsection{Concept Generation}
Discuss the methods used and concepts generated in this section.  You may only want to include figures for a couple of the more interesting or representative concepts.  Make use of the appendix to show additional pictures and describe more concept detail.  



\subsection{Analysis of Alternatives}
Use this section to describe how your team reduced all of your concept alternatives to a smaller subset chosen for further investigation and prototyping.  Make sure to include your ranking criteria along with justification as to why the criteria chosen is important. 





\section{Prototyping Process}
Again include a 4-5 sentence summary of the entire prototyping process including POCs, Alpha, Beta, and Final prototypes.  


\subsection{Proof of Concepts}
Use this section to describe your proof of concepts.  Your team may choose to include 1 image within the text body and use the appendix to show additional material.  Between each subsection within section 4 make sure to describe the progression between concepts.  


\subsection{Intermediate Prototypes}
Use this section to describe your intermediate prototypes.  Make sure to discuss specifically how your prototypes are different from your proof of concepts.  It is important to show the design progression.  Justify your decision in how/why you went about building each of your prototypes.  Tell the reader what you learned from the prototypes.

You may also include notes on how you tested sub-parts of your project, and highlight any design aspects/modifications/issues relating to integration of these sub-parts.  



\subsection{Final prototype}
Describe your final prototype in this section along with results.  Does it meet customer needs, how did this progress from the beta prototype, are there any remaining issues.  Depending on your design progression there may or may not be many changes from the beta prototype - address that here.  If there are no changes, justify your decision to stay with the beta prototype.  If there were many changes make sure to describe them.  




\section{Related design activities}
Again, include a brief introductory summary at the beginning of the section.


\subsection{Risk examination}


\subsection{Skills acquired during this project}



\section{Conclusions and Future Work}
Brief introductory summary.  

\subsection{Conclusions and Project Summary}
Summarize your efforts and results of this project.  

\subsection{Future Work}
In this section discuss any items your team did not complete to your or the evaluators satisfaction.  What things would you do differently? Highlight any tricky issues.  Imagine this as a starting point for another redesign process.  What information would be important to the next design team?  What would you do differently to implement this as a “real” product?  Does your product (or slight modifications of it) have any other applications?  

\newpage
\appendix 

\section{Appendix I}

\section{Appendix II}



\end{document}				
